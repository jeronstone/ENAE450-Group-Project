\documentclass{article}
\usepackage{graphicx} % Required for inserting images
\usepackage{listings} % For code snippets


\title{ENAE450 Final Project Report}
\author{Brandon Newman, Sriman Selvakumaran, Adithya Sundar, Joshua Stone}
\date{Due: May 18th, 2024}

\begin{document}

\maketitle

\section{Introduction}

\section{Simulation}

\subsection{Methods}

\subsection{Results}

\section{Hardware}

\subsection{Methods}

\subsubsection{Navigation with Lidar}

\subsubsection{Aruco Marker Detection}

For Aruco marker detection, our team used the opencv library to analyze the camera frame from the turtlebot. \par

First, we import opencv (\verb|cv2|) and define the aruco detection variables. We also import the \verb|Image| class for the camera subscriber:

\begin{lstlisting}
    import cv2
    from sensor_msgs.msg import Image

    dictionary = cv2.aruco.getPredefinedDictionary(cv2.aruco.DICT_4X4_1000)
    parameters = cv2.aruco.DetectorParameters()
    detector = cv2.aruco.ArucoDetector(dictionary, parameters)
\end{lstlisting}

Next, we created a subscriber to the \verb|/image_raw| topic to get the camera data from the turtlebot:

\begin{lstlisting}
    self.scan_subscriber = self.create_subscription(
        Image,'/image_raw',self.frame_handler,10)
\end{lstlisting}

In the subscriber callback function, we use the open cv library to detect arucos on the frame:

\begin{lstlisting}
    def frame_handler(image_data, self):
        # get camera frame data
        frame = image_data.data

        # detect arucos
        (corners, ids, rejected) = detector.detectMarkers(frame)
\end{lstlisting}

TODO explain how we interfaced this with rest of code

\subsection{Results}

\section{Retrospective}

\subsection{What Went Well}

\subsection{What We Would've Done Differently}

\subsection{Who Did What}

\end{document}
